\documentclass[
	article,			
	11pt,				
	oneside,			
	a4paper,			
	english,			
	brazil,				
	]{abntex2}
	
\usepackage[utf8]{inputenc}	 
\usepackage{amsmath}

\DeclareMathAlphabet{\mathpzc}{OT1}{pzc}{m}{it}

\author{Ruan Costa  nº USP 7990929 e \\ Vinícius Silva nº USP 7557626}
\title{Primeiro EP de MAC239 \\ Resolvedor de Sudoku}

\begin{document}
\maketitle
\tableofcontents
\section{A Lógica do Sudoku}
O primeiro desafio do EP era codificar as regras do Sudoku em cláusulas lógicas. 
Para isso, usamos o modelo do problema das n rainhas.
Naquele caso as regras eram:
\begin{itemize}
\item Tem de haver pelo menos uma rainha em todas as linhas.
\item Pode haver apenas uma rainha em cada linha.
\item Pode haver apenas uma rainha em cada coluna.
\item Pode haver apenas uma rainha em cada diagonal.
\end{itemize}
No caso do Sudoku há a complicação de não podermos trabalhar com apenas dois estados em cada  ``espaço'' (combinação de linha e coluna).
A solução foi tratar cada espaço como um vetor de 9 casos.
Desta forma trabalhamos o sudoku em três dimensões: linha x coluna x número.
Portanto o conjunto dos átomos das cláusulas é definido da seguinte forma:
\begin{displaymath}
C = \{\vec{x} \in N \times N \times N : 1 \leq x_i \leq 9\}
\end{displaymath}
Podemos também definir a função ``ligado'':

\begin{displaymath}
l:N \times N \times N\longrightarrow\{0,1\}
\end{displaymath}

\begin{displaymath}
l(\vec{x}) = 
\begin{cases}
1, \texttt{ se o espaço $(x_1,x_2)$ contém o número em $x_3$}\\
0, \texttt{ se o espaço $(x_1,x_2)$ não contém o número em $x_3$}
\end{cases}
\end{displaymath}
\\
Todos os vetores tais que $l(\vec{x}) = 1$, chamaremos de ligado, e os outros chamaremos de desligado.
Assim, podemos traduzir as regras do Sudoku sa guinte forma:

\begin{itemize}
\item Tem de haver pelo menos um vetor ligado em cada espaço.
\item Pode haver apenas um vetor ligado em cada espaço.
\item Pode haver apenas um vetor ligado em cada coluna com o mesmo número.
\item Pode haver apenas um vetor ligado em cada linha com o mesmo número.
\item Pode haver apenas um vetor ligado em cada ``grid'' com o mesmo número.
\end{itemize}
Se considerarmos cada vetor $\vec{x} \in C$ um átomo, poderiamos escrever essas regras na forma clausal com as seguintes clausulas:

\begin{itemize}
\item $\bigwedge_{k=1}^9\bigwedge_{j=1}^9\bigvee_{i=1}^9 (k,j,i)$  
\item $\bigwedge_{r=1}^9\bigwedge_{k=1}^9\bigwedge_{j=1}^9\bigwedge_{i=j+1}^9 (\neg(r,k,j)\vee\neg(r,k,i))$
\item $\bigwedge_{r=1}^9\bigwedge_{k=1}^9\bigwedge_{j=1}^9\bigwedge_{i=j+1}^9 (\neg(r,j,k)\vee\neg(r,i,k))$
\item $\bigwedge_{r=1}^9\bigwedge_{k=1}^9\bigwedge_{j=1}^9\bigwedge_{i=j+1}^9 (\neg(j,r,k)\vee\neg(i,r,k))$
\item Para cada grid de centro (r,s): $\bigwedge_{k=1}^9\bigwedge_{j=1}^9\bigwedge_{i=j+1}^9 (\neg(j',j'',k) \vee \neg(i',i'',k))$ \\tais que 
$\begin{cases}
j' = (j-1)/3+r-1 \\
j'' = (j-1)\texttt{mod(3)}+s-1 \\
i' = (i-1)/3+r-1 \\
i'' = (i-1)\texttt{mod(3)}+s-1
 \end{cases}$
\end{itemize}
\newpage
\section{Implementação}
Com as cláusulas definidas, o próximo passo foi implementar um programa que escreve o arquivo de entrada para o Minisat.
Escolhemos a linguagem Perl devido à facilidade em manipular strings e arquivos.
Caso o Minisat não esteja na pasta do programa, é preciso mudar a variável ``\$minisat\_path'' com o caminho correto.
A lógica utilizada foi bem simples, o programa segue os seguntes passos:
\begin{enumerate}
\item Abre o arquivo de entrada.
\item Cria uma matriz $9 \times 9 \times 9$ enumerada de 1 a 729.
\item Imprime em um arquivo .cnf os valores correspondentes aos campos fixos do Sudoku.
\item Imprime no mesmo arquivo .cnf as clausulas mapeadas na matriz, usando a lógica da sessão anterior.
\item Chama o Bash para executar o minisat.
\item Imprime a saída: 
\begin{enumerate}
\item Caso não haja solução, imprime "Insatizfazível".
\item Caso haja solução, imprime "Satizfazível" e uma solução.
\end{enumerate}
\end{enumerate}
\end{document}